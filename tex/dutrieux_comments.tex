Line 333: confusing: this affects the baroclinic circulation, hence melt, but not the barotropic circulation
--> but from Jan's 2014 paper: "Since the flow is approximately geostrophic, the increase in zonal
speed can be related to changes in the stratification. In particular, the meridional density gradients steepen, as the mixed layer thins and rises and the pycnocline is pushed upward by the dense bottom layer."

Top of P12: Remember ustar is not the same as u_barotropic, so we need to clarify the importance of the baroclinic component and its connection with the stratification
--> ?

Line 388: What's happening in fig. 5 b and c at 34km retreat? Very low heat content??? 
--> Check that this really is in steady state?

Line 394: Pierre doesn't understand vorticity comment
Physical barrier induces drag right, with provides a source of vorticity that breaks the barotropic picture? Better to say induces viscous stresses?

Line 400: To me, it looks like there is PV leakage created at the ice front such that we go from a closed, melt driven circulation to an open circulation with a much more efficient front exchanges, perhaps paradoxically leading to decreased circulation in the back of the cavity?
--> Isn't that roughly the same thing?

Line 420: Merge fig 4 and 6?
--> could add figure 4b as an inset in 6a?

Lines 441-454
--> Confused/not sure how to fix

Figure 10: Note that the A-B line does not track the more sinuous top of the ridge
--> Should I change this to match figure 1, or just add a comment underneath?

Line 679: e.g. Joughin et al 2016
--> Huh? Not sure why reference is necessary here.


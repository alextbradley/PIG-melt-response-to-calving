- P1: Title: ' quite specified to PIG and not specific to the topic' 
	Suggest changing to \textbf{[On the] Melt Response to Calving Events in Cavities with a Topographic Barrier}? 
	
- P2: "has it been suggested [that future calving events are likely]"? 
    Yes, I think so. Lhermitte et al. 2019 "We hypothesize that the combination of localized ice shelf thinning, enhanced velocity gradients, and the rapid development of the damage areas in the shear zones of PIG’s and TG’s ice shelves is a sign that these ice shelves are already preconditioned for further disintegration". Have left the "significantly reduce" comment, for lack of a better way to say this.
    
- P2: "maybe 'ridge--draft gap' -> 'ridge-crest gap'": not sure that this is necessary, I prefer ridge--draft gap.

- P4: "Thus there is a potential for a feedback involving ice front retreat" --> not sure I agree that this statement follows here? Essentially I'm saying that increased calving might promote acceleration, which should act as a negative feedback (more calving --> advance front), rather than a positive feedback?

- P5 "is this true in the code you are using?  In my recent work gamma would be a linear function of u*.  In an ideal world, that would be the case and you would write Gamma u* (T-Tb) everywhere, where Gamma is an upper case gamma." --> I think I have gamma_T as in the De Rydt 2014 (their equation 5)

- P6: "is it worth offering a narrative for this seemingly unphysical choice?  There is not much justification we can offer, other than ice/ocean boundary layers are very poorly understood.  Maybe it is worth waiting for the reviewers to raise it." --> I wouldn't know how to justify this. Perhaps we could say that results are qualitatively similar if we set this to the same value?

- P9: "this is fine but if you really wanted to be a geek you could average them over the Losch boundary layer instead, since that is the thing that goes into the melt rate formulation" --> how do you define the Losch boundary layer..?
- P1: Title: ' quite specified to PIG and not specific to the topic' 
	Suggest changing to \textbf{[On the] Melt Response to Calving Events in Cavities with a Topographic Barrier}? 
	
- P2: "has it been suggested [that future calving events are likely]"? 
    Yes, I think so. Lhermitte et al. 2019 "We hypothesize that the combination of localized ice shelf thinning, enhanced velocity gradients, and the rapid development of the damage areas in the shear zones of PIG’s and TG’s ice shelves is a sign that these ice shelves are already preconditioned for further disintegration". Have left the "significantly reduce" comment, for lack of a better way to say this.
    
- P2: "maybe 'ridge--draft gap' -> 'ridge-crest gap'": not sure that this is necessary, I prefer ridge--draft gap.

- P4: "Thus there is a potential for a feedback involving ice front retreat" --> not sure I agree that this statement follows here? Essentially I'm saying that increased calving might promote acceleration, which should act as a negative feedback (more calving --> advance front), rather than a positive feedback?

- P5 "is this true in the code you are using?  In my recent work gamma would be a linear function of u*.  In an ideal world, that would be the case and you would write Gamma u* (T-Tb) everywhere, where Gamma is an upper case gamma." --> I think I have gamma_T as in the De Rydt 2014 (their equation 5)

- P6: "is it worth offering a narrative for this seemingly unphysical choice?  There is not much justification we can offer, other than ice/ocean boundary layers are very poorly understood.  Maybe it is worth waiting for the reviewers to raise it." --> I wouldn't know how to justify this. Perhaps we could say that results are qualitatively similar if we set this to the same value?

- P9: "this is fine but if you really wanted to be a geek you could average them over the Losch boundary layer instead, since that is the thing that goes into the melt rate formulation" --> how do you define the Losch boundary layer..?

- P10: "doesn't matter either way of course, but I find it easier to think about the length of the remaining ice shelf - it is nice round numbers and the grounding line feels like a more solid choice of origin than the uncalved ice front - e.g. you set up the ridge to be 50km from the GL, calculate melt rates over 30km from the GL, etc." This is doable, and probably makes sense, but it's useful to have calving proceeding left to right (to match orientation of PIG in figure 1 and 2) --> suggest adding a second axis with shelf length remaining?

- P12: "see comment above

could you define a new thing that is like (5) and (6) but for the product u*(T-Tb)?  Then that should vary like M, if linear." --> suspect I have confused with this comment. I mean that you cannot simply add the velocity and thermal driving effects to get the melt effect. Have removed (commented out) this senstence which I feel is not helpful or adds anything.

- P12: "The PV barrier does not relax until the ice retreats BEYOND the ridge crest. So the change before that point may need explaining.  My guess would be that the ice front induces vorticity in the water column through lateral shear, and so when the ice front is sitting over the ridge, this vorticity source allows the barotropic flow to break the f/h constraint."
I have rewritten this paragraph to reflect this comment, with Paul's interpretation, but I'm not sure how to demonstrate that the "ice front induces vorticity in the water column through lateral shear"

- P17: "The DEM gives the ice freeboard, which is adjusted to ice thickness using assumed ice and seawater densities.  This gives the ice draft, which is compared to Autosub." 
--> Feel like my description captures this, but perhaps this description is ambiguous?

- P20 "which is wind-driven?" 
--> Not quite sure how to parse this comment





%%%%%%%%%%%%%%%%%%%%%%%%%%%%%%%%%%%%%%%%%%%%%%%%%%%%%%%%%%%%%%%%%%%%%%%%%
%%%%%% Paul H Meeting 06-08-2021 %%%%%%%%%%%%%%%%%%%%%%%%%%%%%%%%%%%%%%%%
Realistic plots:
    - We don't need the sections, they're confusing and do not add a lot of info. All we might use them for is to say that on average, the inner cavity temperature is changing (in the north case) or staying the same (in the south cavity); this info can be distilled from the Millgate decompositions.
    - We want plots of streamfunction contours with f/h (colours): show that the spatial pattern of streamfunction is explained by the f/h. SF closely linked to melt.
        --> This could be an important implication of work: showing that PIG circulation is basically explained by $f/h$ contours. 
    - There is a PV barrier between the two regions of inner cavity that you have highlighted. This is why you get the double inner gyre. Make sure this is made explicit. 
    - For your own interest, you should plot the isopycnals through the centre of the gyre seen in scenario 4: how are melting and restoring reconciled before and after snapping beyond the ridge? This balance is different when we have removed a section of $f/h$ space.
    
Idealized plots:
    - Add the 0.1 contour to the outer cavity, showing that there is a circulation there (it is a region of closed f/h). (Regions of closed f/h promote circulation)

Physics: three regions of closed f/h
    - Domain is really three regions of closed f/h contours: inner cavity, outer cavity, and the open ocean. Expect circulation in each, the direction of which depends on the forcing (we see a difference between idealized runs, where cyclonic in open ocean vs anticyclonic in the realistic simulations; suspect that f/h can explain circulation patterns in both cases, so that this difference can be ascribed to complex geometry).
    - Crucially, in both idealized and realistic simulations, we see a qualitative change in the behaviour in the inner cavity when the outer cavity is removed and the domain is reduced to two regions of closed f/h. 
    - In the idealized case, not a lot happens in the outer cavity. "Changing the f/h contours induces a fundamental change in the circulation".
    - A strong hint that f/h dynamics the most important comes from intrusion of anti-cyclonic gyre outside cavity in scenario 4 under southern shear margin. At intrusion, discontinuity in f/h small, so geostrophic flow can penetrate.
    - You see the anti-cyclonic circulation in the outer cavity in the W = 100 case (cyan contour). However, in the wider gap case, the domains aren't quite so neatly separated, and we don't see this circulation.
    
Why is barotropic the thing to look at in the idealized experiments?
    - Approach 1: let's do a thought experiment: assume that flow is purely barotropic and see how far this gets us in explaining the behaviour (a long way!). Flow is indeed largely barotropic, with some leakage across the ridge caused by overturning circulation.
    - Approach 2: when we snap, we change two things: the buoyancy flux, and the water column thickness. The latter can only change the flow through barotropic dynamics, so we should consider the barotropic sf in the baseline, because that is what will change when we calve. 
    - [Our approach is really just considering Navier-Stokes, but we're using a depth averaged form because we're changing the water column thickness, and considering the curl because we're interested in rotation.]

    
